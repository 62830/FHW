\documentclass{fhw}

\usepackage{blindtext}

\title{Beautiful Homework}
\author{Author}
\ID{b11902017}
\CourseID{CSIE5302}
\CourseName{\LaTeX\;Template}

% Toggle this for better BW printing
\NoPrint
% \YesPrint

\begin{document}

\maketitle

\section{Basic Advanced Usage}

\subsection{Enumerate}

Enumerate can be used like this:
\begin{Code}{tex}
\begin{enumerate}[(a)]
  \item one
  \item deux
    \setcounter{enumi}{4}
  \item fünf
    \begin{enumerate}[i.]
      \item and you can
      \item nest them.
    \end{enumerate}
  \item[item6] Custom item.
\end{enumerate}
\end{Code}

The code will become: 

\begin{enumerate}[(a)]
	\item 一
	\item deux
	\setcounter{enumi}{4}
	\item fünf
		\begin{enumerate}[i.]
			\item and you can
			\item nest them.
		\end{enumerate}
	\item[item6] Custom item label.
\end{enumerate}

\subsection{Math symbols}

There are some new symbols defined in this template:

\begin{Code}{tex}
\[
\N \Z \Q \R \C \contra
\abs{a}, \ceil{b}, \floor{c}, \inner{x}, \norm{y}, \set{1, 2, \frac{3}{7}}
a * b = c, x \ast y = z
\]
\end{Code}

\[
\N \Z \Q \R \C \contra
\abs{a}, \ceil{b}, \floor{c}, \inner{x}, \norm{y}, \set{1, 2, \frac{3}{7}}
a * b = c, x \ast y = z
\]

You can also use \verb|*| for product, no more \verb|\cdot|!

If you want to have equations with multiple lines and be aligned, you can use \verb|align*|:

\begin{align*}
  & \int_{0}^{3} \abs{v\left(t\right)} dt \\
  & = \int_{0}^{1} -v\left(t\right) dt + \int_{1}^{3} v\left(t\right) dt \\
  & = \eval{\left(\frac{1}{3}t^3 + \frac{3}{2} t^2 - 4t\right)} _ {1} ^ {0}
  + \eval{\left(\frac{1}{3}t^3 + \frac{3}{2} t^2 - 4t\right)} _ {1} ^ {3} \\
  & = \left[0 - \left(\frac{1}{3} + \frac{3}{2} - 4\right)\right] + \left[\left(9 + \frac{27}{2} - 12\right) - \left(\frac{1}{3} + \frac{3}{2} - 4\right)\right] \\
  & = \frac{89}{6}
\end{align*}

Remember to use \verb|\left( x \right)| for parantheses autoscaling, and \verb|\limits| to put things on top of / under $\sum$ and more! More examples:

\[
  \abs{1 + \frac xy}^2
\]
\[
	\abs{\frac{1}{1 - \lambda h}} \le 1
	\qquad\text{and}\qquad
	\bigcup_{i=1}^n \; \set{z \in \C \mid \abs*{z - a_{ii}} \le {\sum\limits_{j \ne i}} \abs*{a_{ij}}}.
\]

\subsection{Text ornaments}

\begin{Code}{tex}
\textbf{Bold text}, \textit{Italic \footnote{Actually it's oblique.} text} with footnote or \textsl{Slanted text}, \texttt{Teletype text}, \emph{Emphasis with underline}, \underline{Underlined}, \sout{Strike through} work like this.
\end{Code}

\textbf{Bold text}, \textit{Italic \footnote{Actually it's oblique.} text} with footnote or \textsl{Slanted text}, \texttt{Teletype text}, \emph{Emphasis with underline}, \underline{Underlined}, \sout{Strike through} work like this.

\subsection{URLs and hyperrefs}

\begin{Code}{tex}
URLs like this: \url{https://tex.stackexchange.com/questions/61015/how-to-use-different-colors-for-different-href-commands}\\
\href{https://en.wikibooks.org/wiki/LaTeX/Counters}{And hyperref like this.}
\end{Code}
URLs like this: \url{https://tex.stackexchange.com/questions/61015/how-to-use-different-colors-for-different-href-commands}\\
\href{https://en.wikibooks.org/wiki/LaTeX/Counters}{And hyperref like this.}

\section{Blocks}

\subsection{Theorems, Lemmas, ...}

\begin{Code}{tex}
\begin{theorem}
  Theorem 1. \blindtext
\end{theorem}

\begin{lemma}
  Lemma 1. They have seperate numbers.
\end{lemma}

\begin{observation}
  My observation. You can make \verb|\label{}|s...
  \label{myob}
\end{observation}

\begin{lemma}[My lemma]
  ...give them names, and reference them like this:\\
  According to \cref{myob}...
\end{lemma}
\end{Code}

\begin{theorem}
	Theorem 1. \blindtext
\end{theorem}

\begin{lemma}
	Lemma 1. They have seperate numbers.
\end{lemma}

\begin{observation}
  My observation. You can make \verb|\label{}|s...
	\label{myob}
\end{observation}

\begin{lemma}[My lemma]
	...give them names, and reference them like this:\\
	According to \cref{myob}...
\end{lemma}

\subsection{Problems!}

This might be the most useful command of all!
\begin{Code}{tex}
\begin{problem}
  This is a problem.
\end{problem}
\problem You can also use without \verb|\begin| and \verb|\end|.
\problem* If you don't want \verb|\problem| to start on a new page, you can add an asterisk after them. It's unlike other commands, where adding an asterisk disables numbering.
\problem*[C8763 --- Starburst] You can also number them yourself!
\problem* Problem numbering is independent, too.
\end{Code}

\begin{problem}
  This is a problem.
\end{problem}
\problem You can also use without \verb|\begin| and \verb|\end|.
\problem* If you don't want \verb|\problem| to start on a new page, you can add an asterisk after them. It's unlike other commands, where adding an asterisk disables numbering.
\problem*[C8763 --- Starburst] You can also number them yourself!
\problem* Problem numbering is independent, too.

Why do problems start on new page? Because it's more convenient when dealing with GradeScope!

\subsection{Code Blocks}

You can input code files like this:

\begin{Code}{tex}
% \CodeFile[minted options]{lang}{filename}{caption}{label}
\CodeFile[firstline=2]{cpp}{a.cpp}{Example code}{ex}
And refer to them as \cref{code:ex}!
\end{Code}

\CodeFile[firstline=2]{cpp}{a.cpp}{Example code}{ex}
And refer to them as \cref{code:ex}!

Or write code inside the tex file like this:
\begin{Code}{tex}
\begin{Code}{cpp}
#include <iostream>
int main() {}
\end{Codee} % extra e cause it won't compile
\end{Code}

\begin{Code}{cpp}
#include <iostream>
int main() {}
\end{Code}

% This is an inline code test: \InlineCode{#include <iostream>}
\mintinline{cpp}{for (int i = 0; i < 'a'; ++i) cout << zisk; }

\subsection{Images}

Though you can manually add images by \verb|\includegraphics|, I've made a command here:
\begin{Code}{tex}
% \Image[size in \textwidth's default 0.8]{filename}{caption}{label}
\Image[0.7]{cactus.jpeg}{Cactus!}{cactus}
And refer to them like \cref{img:cactus}.
\end{Code}

% \Image[size in \textwidth's default 0.8]{filename}{caption}{label}
\Image[0.7]{cactus.jpeg}{Cactus!}{cactus}
And refer to them like \cref{img:cactus}.

\section{TikZ}

I don't know how to teach TikZ in a short-enough-to-fit-in-this-document length, so here's one example:

\CodeFile{tex}{omega.tex}{TikZ example}{tikzex}

It looks like this:

\Image{omega.pdf}{TikZ example result}{tikzexr}

I always compile them as different files and input them with figures. However, they can also be done inside the same tex file.

\section{Lorem Ipsums}

\subsection {Chinese lorem}

\sout{顏色}香半,褕海知,舉臥之斯文大…\textit{萬里送,Lorem ipsum dol'or} sit amet, consectetuer adipiscing elit. 幾度九華帳山色之難難,楚能余亦之難難夜日行陽為君。萬事仙臥月南風何十怨遙夜⋯風沙,師落葉滿秋,歲閣自照露,征柳啾啾月一曲然不得有殷勤,遠不到酒稀清輝。腸何時還,行路難獨夜南⋯羅微鳳路雪虛征戰。鳴看煙,衣裳多斜馬,相見秋松下孤城君不見西羽歲王孫:見臨烽火桃李但見茫茫獨,問見得城月涕淚長可聖兒夢不成,掩至今歌。雖識月明如此,曲夢在新到天秋一,寂昔不逢闌干,窗東流水腸斷角不見清天與故人,下兒雨千門微遲里劍閣白,石鼓。

\blindtext

\subsection{Ipsum}

\blindtext

沒有好棒,別嗚哈哈哈,都這個實說不然也不,的人的什麼最好。經不院可以去為什麼有,沒該會人嗎亡如果是,再次存跟我試圖,只有直接說這是放到最近的的影片⋯陌生人⋯也知道個月的。其覺得是起來跟不可能也很,畫面待喔好天的簡單很不是聽起來,然已經就聖我的話的聲。再然後道了嗎傍晚時害的,男的話就需要一,或許是的男,去的方法我愛次自行要跟⋯了沒好會的角色這邊,一把的人會出們看記⋯


\end{document}
