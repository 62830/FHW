\documentclass{fhw}

\usepackage{blindtext}

\title{Beautiful Homework}
\author{Author}
\ID{b11902000}
\CourseID{Math4007-01}
\CourseName{Calculus 2}

\begin{document}

\maketitle

\blindtext
\Code[firstline=2]{cpp}{a.cpp}{caption}{label}
\blindtext
\Image{cactus.jpeg}{Cactus!}{cactus}

\section{Lorem}

Zisk \footnote{Known as ZCKevin} is strong. \cref{code:label} and \cref{img:cactus}.

\subsection {Chinese lorem}

\sout{顏色}香半,褕海知,舉臥之斯文大…\textit{萬里送,Lorem ipsum dol'or} sit amet, consectetuer adipiscing elit. 幾度九華帳山色之難難,楚能余亦之難難夜日行陽為君。萬事仙臥月南風何十怨遙夜⋯風沙,師落葉滿秋,歲閣自照露,征柳啾啾月一曲然不得有殷勤,遠不到酒稀清輝。腸何時還,行路難獨夜南⋯羅微鳳路雪虛征戰。鳴看煙,衣裳多斜馬,相見秋松下孤城君不見西羽歲王孫:見臨烽火桃李但見茫茫獨,問見得城月涕淚長可聖兒夢不成,掩至今歌。雖識月明如此,曲夢在新到天秋一,寂昔不逢闌干,窗東流水腸斷角不見清天與故人,下兒雨千門微遲里劍閣白,石鼓。


Some words imediatly after code block\ldots

$$
\abs{1 + \frac xy}^2
$$

\[
	\abs{\frac{1}{1 - \lambda h}} \le 1
	\qquad\text{and}\qquad
	\bigcup_{i=1}^n \; \set{z \in \C \mid \abs*{z - a_{ii}} \le {\sum\nolimits_{j \ne i}} \abs*{a_{ij}}}.
\]

\blindtext

\newpage

\section{Ipsum}
\blindtext

沒有好棒,別嗚哈哈哈,都這個實說不然也不,的人的什麼最好。經不院可以去為什麼有,沒該會人嗎亡如果是,再次存跟我試圖,只有直接說這是放到最近的的影片⋯陌生人⋯也知道個月的。其覺得是起來跟不可能也很,畫面待喔好天的簡單很不是聽起來,然已經就聖我的話的聲。再然後道了嗎傍晚時害的,男的話就需要一,或許是的男,去的方法我愛次自行要跟⋯了沒好會的角色這邊,一把的人會出們看記⋯

\problem*
\problem*
\problem*
\section{test}
\problem*
\blindtext
\problem*[8]
\subsection{test}
\problem*

\begin{theorem}
	Theorem 1. \blindtext
\end{theorem}

\begin{lemma}
	Lemma 1.
\end{lemma}

\begin{observation}
	My observation.
	\label{myob}
\end{observation}

\begin{lemma}[My lemma]
	My lemma.
	According to \cref{myob}.
\end{lemma}

\begin{theorem}[My theorem]
	Another theorem
\end{theorem}

\begin{lemma}
	Lemma 2.
\end{lemma}

Teeeeest url: \url{https://tex.stackexchange.com/questions/61015/how-to-use-different-colors-for-different-href-commands}
\href{https://en.wikibooks.org/wiki/LaTeX/Counters}{Test hrefy}

\end{document}
